\documentclass[article,12pt,twocolumn]{IEEEtran}
\usepackage{amsmath}
\usepackage{tfrupee}
 
\begin{document} 

\title{Assignment 1}
\author{Rudransh Mishra, AI21BTECH11025}
\date{April 2022}
\maketitle

\textbf{ICSE 10th, 2014 paper}


\hfill

\textbf{Question 1(b)} 
Ranbir borrows \rupee 20,000 at 12 \% compound interest. If he repays \rupee 8400 at the end of the first year and \rupee 9680 at the end of the second year, find the amount of loan outstanding at the beginning of the third year.

\hfill \break
\textbf{Solution:}

Initial loan taken by Ranbir, $P$ = \rupee $20000$

Interest rate, $I$ = $12 $ $\%$

Time between compunding, $T$ = $1 year $

\hfill

Amount in compound interest, 

\hfill

$A$ = $ P * (1+I/100)^T $ 

\hfill

(Value of I is in percent)

\hfill

Therefore, amount due at the end of one year is 

\hfill

$A$ = $20000 * (1+12/100)^1 $

$A$ = \rupee $22400$

\hfill

Amount paid at the end of one year is \rupee 8400.

Thus remaining amount,

\hfill

$A$ = $22400-8400$

$A$ = \rupee $14000$

\hfill

Now, amount due at the end of the second year is 

\hfill

$A$ = $14000 * (1+12/100)^1 $

$A$ = \rupee $15680$

\hfill

Amount paid at the end of second year is \rupee 9680.


Thus remaining amount,

\hfill

$A$ = $15680-9680$

$A$ = \rupee $6000$

\hfill

\hfill

\hfill

\hfill

Therefore, we know that he still has \rupee $6000$ to pay to the bank.

\hfill

$\implies$ Ranbir still owes to the bank \rupee $6000$ out of the \rupee $20,000$ he had borrowed, after the two annual payments.


\end{document}
