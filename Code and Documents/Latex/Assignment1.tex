\documentclass[article,12pt,twocolumn]{IEEEtran}
\usepackage[utf8]{inputenc}
\usepackage{mathtools}
\usepackage[a4paper, total={6in, 8in}, margin = 1in]{geometry}
\usepackage{enumitem}
\usepackage{graphicx}
\graphicspath{{images/}}
\usepackage{amsmath}
\usepackage{tfrupee}
\newcommand{\myvec}[1]{\ensuremath{\begin{pmatrix}#1\end{pmatrix}}}
\let\vec\mathbf
\let\orupee\rupee
\def\rupee{\ifmmode\text{\orupee}\else\orupee\fi}
\title{Assignment 1}
\author{Rudransh Mishra, AI21BTECH11025}
\begin{document}
\maketitle
\section*{Question 1a}
\noindent Ranbir borrows \rupee 20,000 at 12 \% compound interest. If he repays \rupee 8400 at the end of the first year and \rupee 9680 at the end of the second year, find the amount of loan outstanding at the beginning of the third year.
\section*{Solution}
\noindent Initial loan taken by Ranbir, \\
P& = \rupee 20000\\
Interest rate, I & = 12 \% \\
Time between compounding, T& = 1 year\\
\begin{align*}
A& =  P \times (1+I/100)^T
\end{align*}
Where value of I is in percent.\\
\\
Therefore, amount due at the end of one year is
\begin{align*}
A & = 20000 \times (1+12/100)^1 \\
A & = \rupee 22400
\end{align*}
Amount paid at the end of one year is \rupee 8400.\\
Thus remaining amount,
\begin{align*}
A& = 22400-8400 \\
A& = \rupee 14000
\end{align*}
\noindent This new amount will now be the principal amount for the next year.\\
Thus, amount due at the end of the second year is 
\begin{align*}
A& = 14000 \times (1+12/100)^1\\
A& = \rupee 15680
\end{align*}
Amount paid at the end of second year is \rupee 9680.\\
Thus remaining amount,
\begin{align*}
A& = 15680-9680\\
A& = \rupee 6000
\end{align*}
Therefore, we know that he still has \rupee $6000$ to pay to the bank.\\

$\implies$ Ranbir still owes to the bank \rupee $6000$ out of the \rupee $20,000$ he had borrowed, after the two annual payments.
\end{document}
